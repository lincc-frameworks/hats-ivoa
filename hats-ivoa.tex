\documentclass[11pt,a4paper]{ivoa}

\usepackage{xcolor}  % for colored text
\usepackage{longtable}
\usepackage{array}
\usepackage{colortbl}

\usepackage{listings}
\lstset{ %
  backgroundcolor=\color{white},   % choose the background color; you must add \usepackage{color} or \usepackage{xcolor}
  basicstyle=\footnotesize\ttfamily,        % the size of the fonts that are used for the code
  breakatwhitespace=false,         % sets if automatic breaks should only happen at whitespace
  breaklines=true,                 % sets automatic line breaking
  captionpos=b,                    % sets the caption-position to bottom
  frame=single,                    % adds a frame around the code
  keepspaces=true,                 % keeps spaces in text, useful for keeping indentation of code (possibly needs columns=flexible)
  language={},                 % the language of the code
  otherkeywords={*,...},            % if you want to add more keywords to the set
  numbers=none,                    % where to put the line-numbers; possible values are (none, left, right)
  rulecolor=\color{black},         % if not set, the frame-color may be changed on line-breaks within not-black text (e.g.comments (green here))
  showspaces=false,                % show spaces everywhere adding particular underscores; it overrides 'showstringspaces'
  showstringspaces=false,          % underline spaces within strings only
  showtabs=false,                  % show tabs within strings adding particular underscores
  tabsize=2,                       % sets default tabsize to 2 spaces
  title=\lstname                   % show the filename of files included with \lstinputlisting; also try caption instead of title
}

\input tthdefs
\input gitmeta

\title{HATS: A Standard for the Hierarchical Adaptive Tiling Scheme in the Virtual Observatory}

% Borrowed from UCDList.tex
% having descriptions in a narrow column is painful -- the following
% is an attempt to define a suitable column style ("D"escription)
\newcolumntype{D}[1]{>{\raggedright\tolerance=7000\let\newline\\\arraybackslash
  \hspace{0pt}}p{#1}}

\ivoagroup{Applications Group}

\author[https://www.ivoa.net/authors/caplar]{Neven Caplar}
\author[https://www.ivoa.net/authors/usher]{Melissa DeLucchi}
\author[https://www.ivoa.net/authors/caplar]{Sean McGuire}
\author[https://www.ivoa.net/authors/usher]{Sandro Campos}
\author[https://www.ivoa.net/authors/caplar]{Derek Jones}
\author[https://www.ivoa.net/authors/usher]{Doug Branton}
\author[https://www.ivoa.net/authors/offline]{LINCC team...}

\editor{editor here}

\previousversion{This is the first public release}

\begin{document}
\begin{abstract}
    The increasing complexity and volume of astronomical datasets necessitate efficient spatial indexing and query strategies within the Virtual Observatory (VO). The Hierarchical Adaptive Tiling Scheme (HATS) is a framework designed to facilitate scalable queries, filtering operations, and efficient data retrieval across large astronomical surveys. Traditional spatial indexing methods often struggle with the massive scale of modern astronomical datasets, leading to inefficient query execution and storage overhead. HATS provides a flexible, hierarchical approach that balances computational efficiency and adaptability to non-uniform data distributions.

    This document describes the structure, implementation, and best practices for integrating HATS within the VO ecosystem, ensuring interoperability and performance optimization for distributed astronomical datasets. The reference implementation of HATS can be found at \url{https://github.com/astronomy-commons/hats}. Additionally, we discuss how HATS enhances existing indexing schemes, its role in federated data access, and its potential applications for time-domain astronomy, large-scale surveys, and cross-matching of astronomical catalogs.
\end{abstract}

\section*{Acknowledgments}
    The authors thank the IVOA Applications Working Group and various contributors from the astronomical community for their feedback and discussions that shaped this standard. We acknowledge the key collaborations with Space Telescope Science Institute (STScI) and IPAC, whose expertise and contributions have been invaluable in refining the HATS framework. Additionally, we extend our gratitude to  the Strasbourg astronomical Data Center for their assistance in metadata structuring and interoperability support. This work has also benefited from insights provided by the S-Plus survey, Linea, European Space Agency, and Canadian astronomy data center, whose contributions have helped enhance the applicability and robustness of HATS in large-scale astronomical data analysis.


\section*{Conformance-related definitions}
The words ``MUST'', ``SHALL'', ``SHOULD'', ``MAY'', ``RECOMMENDED'', and
``OPTIONAL'' (in upper or lower case) used in this document are to be
interpreted as described in IETF standard RFC2119 \citep{std:RFC2119}.

The \emph{Virtual Observatory (VO)} is a
general term for a collection of federated resources that can be used
to conduct astronomical research, education, and outreach.
The \href{https://www.ivoa.net}{International
Virtual Observatory Alliance (IVOA)} is a global
collaboration of separately funded projects to develop standards and
infrastructure that enable VO applications.

\section{Introduction}
    The rapid expansion of astronomical data from large survey facilities like Vera C. Rubin Observatory, Euclid, and Roman Space Telescope necessitates innovative solutions for spatial indexing and efficient data retrieval. These surveys generate vast amounts of high-resolution imaging and time-domain data, requiring efficient methods for organizing, querying, and cross-matching data across multiple archives. Traditional approaches to spatial indexing, such as hierarchical pixelization (e.g., HEALPix) or static tiling schemes, often exhibit inefficiencies when handling dynamic, multi-resolution datasets.

    The Hierarchical Adaptive Tiling Scheme (HATS) is a novel approach designed to optimize spatial data partitioning while maintaining flexibility in accommodating varying data densities. Unlike fixed spatial partitioning methods, HATS dynamically adjusts tile sizes based on local data characteristics, ensuring an optimal balance between resolution, query efficiency, and storage management. By leveraging a hierarchical structure, HATS allows users to perform efficient multi-resolution queries while preserving high precision in regions of interest.

    This document aims to define best practices for implementing and utilizing this tiling scheme within the Virtual Observatory framework. This document outlines the principles behind HATS, describes its data model, and provides recommendations. Additionally, we discuss how HATS can facilitate efficient cross-matching of astronomical catalogs, accelerate large-scale spatial queries, and enhance interoperability between diverse astronomical datasets.


\section{Motivation and Goals}
    The primary motivation behind HATS is to address the following challenges in astronomical data management:
    \begin{itemize}
        \item \textbf{Scalability:} Modern astronomical surveys generate petabytes of spatially distributed data, requiring an indexing scheme that scales efficiently with dataset size.
        \item \textbf{Adaptive Resolution:} Fixed grid-based partitioning often leads to inefficient storage and query execution, particularly in non-uniformly distributed datasets. HATS dynamically adjusts tile sizes to accommodate varying data densities.
        \item \textbf{Efficient Query Execution:} Spatial queries such as nearest-neighbor searches and cross-matching must be executed efficiently across distributed data repositories. HATS enables rapid indexing and retrieval of relevant data subsets.
        \item \textbf{Interoperability:} Astronomical data is collected from diverse instruments and observatories, often using different spatial reference frames. HATS provides a standardized framework for integrating and harmonizing spatial data across multiple sources.
    \end{itemize}

\section{HATS Design and Implementation}

\subsection{HATS Catalog directory structure} \label{sec:catalog}

The HATS framework relies on spatially sharding catalogs 
of approximately the same size in Parquet files. Here, we discuss how this is achieved and
additional concepts that make it easier to use this main idea for astronomical
research.

The central unit of data storage is HATS \texttt{catalog}. 
It stores the data along with the associated metadata needed to access it. 

Focusing on an individual  \texttt{catalog}, the  \texttt{catalog} organization structure is shown in Listing~\ref{fig:exampleCatalogStructure}:

\begin{minipage}{\linewidth}
\begin{lstlisting}[caption=Example catalog directory contents, label=fig:exampleCatalogStructure]
catalog/
|-- [REQUIRED] properties
|-- [RECOMMENDED] partition_info.csv
|-- [OPTIONAL] point_map.fits
|-- [OPTIONAL] data_thumbnail.parquet
+-- dataset/
    |-- [RECOMMENDED] _metadata
    |-- [RECOMMENDED] _common_metadata
    |-- Norder=0/
    |-- Norder=1/
    |-- Norder=2/
    |-- Norder=3/
    |-- Norder=4/
    |-- Norder=5/
    |-- Norder=6/
    +-- Norder=. . ./
\end{lstlisting}
\end{minipage}

The astronomy data is stored in the directory \texttt{dataset}, within the subdirectories that specify the order at which particular part of the dataset is stored. 
We will discuss the the partitioning and data storage in Sections \ref{sec:hierarchical}, \ref{sec:adaptive} and \ref{sec:parquet}. 
The other files visible above are various metadata and auxiliary files that are here to enable better and easier handling of the data and we will describe them in Section \ref{sec:meta}. 
    
\subsubsection{Hierarchical Structure} \label{sec:hierarchical}
Focusing now on the dataset's contents, HATS employs a multi-level hierarchy based on HEALPix tiling, where each level represents a progressively finer spatial resolution.

All tiles of the same HEALPix order are contained within the same prefix \texttt{Norder=k} directory. 
To avoid directories becoming too large for some file systems, the tiles are then grouped by a \texttt{Dir} subdirectory prefix,
where the value of the \texttt{Dir} key is the result of integer divisiion by 10000 of the pixel number.

We see the directory structure in Listing~\ref{fig:datasetWithLeaf}, showing a dataset with leaf parquet files at several HEALPix orders.

\begin{minipage}{\linewidth}
\begin{lstlisting}[caption=Example catalog dataset directory contents, label=fig:datasetWithLeaf]
dataset/
|-- . . .
|-- Norder=6/
|   |-- Dir=0/
|   |   |-- Npix=0.parquet
|   |   |-- . . .
|   |   +-- Npix=9999.parquet
|   +-- Dir=10000/
|       +-- Npix=10000.parquet
|           +-- . . .
|-- Norder=7/
+-- . . .
\end{lstlisting} 
\end{minipage}

The data is stored in the parquet files (discussed in Section \ref{sec:parquet}), with one or multiple files being possible in the final directory, i.e., in the ultimate data leaf.  \par 
If there are multiple files, they should be read together, i.e., we consider them to be one single data unit. 
In this way, small updates can be added to already existing  \texttt{catalogs} with simple, correctly placed, additions of files in existing folders.

Such a directory structure would appear as shown in Listing~\ref{fig:datasetWithDir}

\begin{minipage}{\linewidth}
\begin{lstlisting}[caption=Example catalog dataset directory contents with leaf directories, label=fig:datasetWithDir]
dataset/
|-- . . .
|-- Norder=6/
|   |-- Dir=0/
|   |   |-- Npix=0/
|   |   |   |-- part0.parquet
|   |   |   +-- . . .
|   |   +-- Npix=9999/
|   |       |-- part0.parquet
|   |       +-- . . .
|   +-- Dir=10000/
|       +-- Npix=10000/
|           |-- part0.parquet
|           +-- . . .
|-- Norder=7/
+-- . . .
\end{lstlisting} 
\end{minipage} 

    \subsubsection{Adaptive Tiling Algorithm} \label{sec:adaptive}
    Unlike static partitioning schemes, HATS dynamically subdivides spatial regions based on data density. In areas with sparse data, larger tiles minimize storage overhead, whereas high-density areas are subdivided into smaller tiles to improve query efficiency. \par
	The data is stored at a given level until the dataset size crosses a predetermined threshold. This threshold can be, most commonly, the number of rows or the size of the data on the disk. At this point, the data gets split into four higher-order HEALPix tiles using the spatial information contained in the data. This process continues until all of the data is stored at the appropriate level and no data leaf has more data than the predetermined threshold.

     \subsubsection{Structure of files} \label{sec:parquet}
The astronomical data is stored in *.parquet format. Parquet is a columnar storage file format optimized for efficient data compression and retrieval, especially well-suited for analytical workloads. It is ideal for storing large amounts of astronomical tabular data because it allows fast access to specific columns without reading the entire dataset, significantly reducing I/O and improving performance. \par
The HATS format RECOMMENDS that the index column of the dataset is \texttt{healpix\_29} index column.  \texttt{healpix\_29} stores the crucial spatial information about the position in the sky for each row, and it is not unique.
This is calculated as the HEALPix order 29 value of the row's right ascension and decliation. If two objects occur at the same location, or the data is individual observations of the same sky object, then multiple rows may have the same value for the \texttt{healpix\_29} column.
The existence of this value speeds up downstream spatial calculations, and is beneficial for spatially-intensive applications.

\subsection{Supplemental Tables} \label{sec:supplemental}

\subsubsection{Margin cache} \label{sec:margin}
\subsubsection{Index tables} \label{sec:index}
\subsubsection{Catalog collection} \label{sec:collection}

Margins and indexes are associated with a single astronomical dataset, and to make this connection clearer and enable friendlier application behavior, tables MAY be grouped together under a single directory called a catalog collection.
These include the primary data \texttt{catalog} and other \texttt{catalogs} that are optional and are intended either to improve access to the main \texttt{catalog} or to enrich it with additional information. 
For instance, one common such \texttt{catalog} is the margin \texttt{catalog}, containing information about objects near the spatial border of each spatial shard. This dramatically simplifies and improves crossmatching ability when using HATS datasets. \par 

\textcolor{red}{TODO, NEED HELP -  Explain the structure of the collection.properties. Explain what the format is, what is contained, what is the obligatory and what is optional information}
In Listing~\ref{fig:exampleCollectionStructure}, we present an overview of this folder structure, including a few common examples of such optional catalogs.
The catalog collection directory MUST contain a \texttt{collection.properties} file, whose content is outlined in Section~\ref{sec:collectionProperties}.

\begin{minipage}{\linewidth}
\begin{lstlisting}[caption=Example collection directory contents, label=fig:exampleCollectionStructure]
gaia_dr3/
|-- collection.properties
|-- catalog/
|   |-- properties
|   +-- . . .
|-- [OPTIONAL] margin_5arcs/
|   |-- properties
|   +-- . . .
|-- [OPTIONAL] margin_10arcs/
|   |-- properties
|   +-- . . .
+-- [OPTIONAL] index_designation/
    |-- properties
    +-- . . .
\end{lstlisting}
\end{minipage}

    \subsection{Metadata and Auxiliary Files} \label{sec:meta}
    HATS implementations utilize auxiliary files and metadata files to store relevant information about the  structure, including:
    \begin{itemize}
    	\item \textbf{[REQUIRED] collection.properties}, at  \texttt{catalog} collection level
        \item \textbf{[RECOMMENDED] partition\_info.csv}, at  \texttt{catalog} level
        \item \textbf{[OPTIONAL] point\_map.fits}, at  \texttt{catalog} level
        \item \textbf{[OPTIONAL] data\_thumbnail.parquet}, at  \texttt{catalog} level
        \item \textbf{[REQUIRED] properties}, at  \texttt{catalog} level
        \item \textbf{[RECOMMENDED] \_metadata}, at  \texttt{catalog}/dataset level
        \item \textbf{[RECOMMENDED] \_common\_metadata}, at  \texttt{catalog}/dataset level
    \end{itemize}
    
    \subsubsection{properties} 

A text file named \texttt{properties} is REQUIRED in the root level of the catalog directory.
It marks the directory as containing a HATS catalog collection, and so MUST be located in the 
root of the catalog collection.
It MUST be encoded in UTF-8, with one line per property, following the syntax \texttt{keyword = value}.
The ordering of the keywords is not important. The keywords MAY include many of those listed in Table~\ref{tab:properties}.

We enforce additional requirements for the presence of particular fields for different types of HATS tables. 
A matrix of these requirements is shown in Table~\ref{tab:propertyRequirements}.


\footnotesize\begin{longtable}[h!]{D{0.33\textwidth} D{0.67\textwidth}}
\sptablerule
\textbf{HATS Keyword}&\textbf{Description - Format - Example}\\
\sptablerule
\endhead

addendum\_did &If content has been added after initial catalog creation, creator\_did of any added data \\
all\_indexes &For catalog collections, space-delimited map of indexed field to subdirectories containing index tables. \\
all\_margins &For catalog collections, space-delimited list of subdirectories containing margin caches. \\
bib\_reference &Bibliographic reference \\
bib\_reference\_url &URL to bibliographic reference \\
creator\_did &Unique ID of the HATS - Format: IVOID - Ex : ivo://CDS/P/2MASS/J \\
data\_ucd &UCD describing data contents \\
dataproduct\_type &Format: one word ONE OF(object, nested, margin, association, index, ...) \\
default\_index &For catalog collections, the field of the default index to use for ID searches. \\
default\_margin &For catalog collections, the subdirectory containing the default margin cache to use for crossmatching \\
hats\_assn\_join\_table\_url &For association tables (i.e. dataproduct\_type == "association"), there will be a join table with original survey data (right side of the join) \\
hats\_assn\_leaf\_files &For association tables (i.e. dataproduct\_type == "association"), does the table contain leaf files (may optionally only provide a "soft" association between tiles only). \\
hats\_builder &Name and version of the tool used for building the HATS – Format: free text -- Example "hats-import v0.6.4" \\
hats\_col\_assn\_join &For association tables (i.e. dataproduct\_type == "association"),column name for the joining (right) side of the join within the original table \\
hats\_col\_assn\_join\_assn &For association tables (i.e. dataproduct\_type == "association"), column name in the assocation table for the join (right) side of the association table. \\
hats\_col\_assn\_primary &For association tables (i.e. dataproduct\_type == "association"), column name for the primary (left) side of the join \\
hats\_col\_assn\_primary\_assn &For association tables (i.e. dataproduct\_type == "association") column name in the association table that matches the primary (left) side of the join \\
hats\_col\_dec &Column name of the dec coordinate. Used for partitioning and default cross-matching. \\
hats\_col\_ra &Column name of the ra coordinate. Used for partitioning and default cross-matching. \\
hats\_cols\_default &Which columns should be read from parquet files, when user doesn't otherwise specify. Useful for wide tables. Format blank separated column names \\
hats\_cols\_sort &At catalog creation time, the columns used to sort the data, in addition to \texttt{healpix\_29} column. \\
hats\_cols\_survey\_id &The primary key used in the original survey data. May be multiple columns if the survey uses a composite key (e.g. object ID and MJD for detections) \\
hats\_coordinate\_epoch &For the default ra and dec (hats\_col\_ra, hats\_col\_dec), the measurement epoch \\
hats\_copyright &Copyright mention associated to the HATS - Format: free text \\
hats\_creation\_date &HATS first creation date - Format: ISO 8601 => YYYY-mm-ddTHH:MMZ \\
hats\_creator &Institute or person who built the HATS – Format: free text – Ex : CDS (T.Boch) \\
hats\_estsize &HATS size estimation – Format: positive integer – Unit : KB \\
hats\_frame &Coordinate frame reference – Format: word “equatorial” (ICRS), “galactic”, “ecliptic” \\
hats\_index\_column &For index tables (i.e. dataproduct\_type == "index"), the column that is indexed over \\
hats\_index\_extra\_column &For index tables (i.e. dataproduct\_type == "index"), extra columns that are carried through with the index \\
hats\_margin\_threshold &For margin tables, the threshold used for finding points within margin. Units: arcs \\
hats\_max\_rows &At catalog creation time, the maximum number of rows per file before breaking into 4 new files at higher order. \\
hats\_nrows &Number of rows of the HATS catalog – Format: positive integer \\
hats\_order &Deepest HATS order – Format: positive integer \\
hats\_primary\_table\_url &For supplemental tables (i.e. dataproduct\_type <> "object"), there will be a primary table with original survey data. \\
hats\_progenitor\_url &URL to an associated progenitor HATS \\
hats\_release\_date &Last HATS update date - Format: ISO 8601 => YYYY-mm-ddTHH:MMZ \\
hats\_service\_url &HATS access url – Format: URL \\
hats\_status &HATS status – Format: list of blank separated words (private” or “public”), (“main”, “mirror”, or “partial”), (“clonable”, “unclonable” or “clonableOnce”) – Default : public main clonableOnce \\
hats\_version &Number of HATS version – Format: 0.1 (corresponds to this specification document) \\
moc\_sky\_fraction &Fraction of the sky covers by the MOC associated to the HATS – Format: real between 0 and 1 \\
npix\_suffix & String to indicate file suffix for leaf files. In the typical HATS directory structure, this is \texttt{'.parquet'} or \texttt{'.pq'} because there is a single file in each Npix partition. If using leaf directories, \texttt{'/'}. \\
obs\_ack &Acknowledgment mention. \\
obs\_collection &Short name of original data set – Format: one word – Ex : 2MASS \\
obs\_copyright &Copyright mention associated to the original data – Format: free text \\
obs\_copyright\_url &URL to a copyright mention \\
obs\_description &Data set description – Format: free text, longer free text description of the dataset \\
obs\_regime &General wavelength – Format: word: "Radio" | "Millimeter" | "Infrared" | "Optical" | "UV" | "EUV" | "X-ray" | "Gamma-ray" \\
obs\_title &Data set title – Format: free text, one line – Ex : HST F110W observations \\
prov\_progenitor &Provenance of the original data – Format: free text \\
publisher\_id &Unique ID of the HATS publisher – Format: IVOID - Ex : ivo://CDS \\
t\_max &Stop time of the observations – Format: real – Representation: MJD \\
t\_min &Start time of the observations – Format: real – Representation: MJD1 \\
\sptablerule    
\caption{Available keys for properties file}
\label{tab:properties}
\end{longtable}

\begin{table}[h!]
\rowcolors{2}{white}{gray!20}
\footnotesize\begin{tabular}{l c c c c c}
\sptablerule
\textbf{HATS Keyword} &\multicolumn{5}{c}{\textbf{HATS Catalog Type}} \\
\rowcolor{white}
& object & margin & index & association  & collection \\
\sptablerule
all\_indexes & & & & &opt \\
all\_margins & & & & &opt \\
dataproduct\_type &REQ &REQ &REQ &REQ & \\
default\_index & & & & &opt \\
default\_margin & & & & &opt \\
hats\_assn\_join\_table\_url & & & &REQ & \\
hats\_assn\_leaf\_files & & & &REQ & \\
hats\_col\_assn\_join & & & &REQ & \\
hats\_col\_assn\_join\_assn & & & &opt & \\
hats\_col\_assn\_primary & & & &REQ & \\
hats\_col\_assn\_primary\_assn & & & &opt & \\
hats\_col\_dec &REQ &opt & & & \\
hats\_col\_ra &REQ &opt & & & \\
hats\_cols\_default &opt &opt & & & \\
hats\_index\_column & & &REQ & & \\
hats\_index\_extra\_column & & &opt & & \\
hats\_margin\_threshold & &REQ & & & \\
hats\_npix\_suffix &opt &opt &opt &opt & \\
hats\_nrows &REQ &REQ &REQ &REQ & \\
hats\_primary\_table\_url & &REQ &REQ &REQ &REQ \\
obs\_collection &REQ &REQ &REQ &REQ &REQ \\
\sptablerule
\end{tabular}
\caption{Catalog-type specific fields. For display, REQ is REQUIRED, and opt is OPTIONAL}
\label{tab:propertyRequirements}
\end{table}



The text file may contain comment lines, beginning with the \texttt{'\#'} character.

An example \texttt{properties} file is shown in Listing~\ref{fig:examplePropertiesFile}.

\begin{minipage}{\linewidth}
\begin{lstlisting}[caption=Example \texttt{properties} file contents, label=fig:examplePropertiesFile]
#HATS catalog
obs_collection=euclid_q1_merFinalCatalog
dataproduct_type=object
hats_nrows=29767806
hats_col_ra=RIGHT_ASCENSION
hats_col_dec=DECLINATION
hats_cols_sort=OBJECT_ID
hats_max_rows=1000000
hats_order=6
moc_sky_fraction=0.00618
hats_builder=hats-import v0.4.4
hats_creation_date=2025-03-20T03\:38UTC
hats_estsize=23137775
hats_release_date=2024-09-18
hats_version=v0.1
\end{lstlisting}
\end{minipage}




\subsubsection{partition\_info.csv} 


    \textcolor{red}{some more melissa text starts}
A text file named \texttt{partition\_info.csv} is OPTIONAL and RECOMMENDED in the root level of the catalog directory.
If present, it MUST be a CSV (comma-separated-values) file, with the columns \texttt{"Norder"} and \texttt{"Npix"}, as shown in example contents in Listing~\ref{fig:examplePartitionInfoCsv}.
Additional columns might be present in the file, but are not required, and may not be interpreted by all HATS readers.
The values of pairs of \texttt{"Norder"} and \texttt{"Npix"} reflect the HEALPix tiles of the catalog's partitions. 
HATS readers can quickly read this file to understand the full scope of the catalog, and potential spatial overlap with other catalogs.


\begin{minipage}{\linewidth}
\begin{lstlisting}[caption=Example \texttt{partition\_info.csv} file contents, label=fig:examplePartitionInfoCsv]    
Norder,	Npix
3,	530
4,	637
4,	958
4,	1003
4,	2147
\end{lstlisting}
\end{minipage}

\subsubsection{point\_map.fits} 

A FITS file containing the two-dimensional histogram of points in each HEALPix tile at some reasonably high order.
This will be at the highest order calculated during the catalog ingestion process (likely 9 or 10). 
This data is useful when inspecting catalogs and understanding the distribution of data. 

This file is OPTIONAL and RECOMMENDED for object catalogs.
           \subsubsection{data\_thumbnail.parquet} 
  This is a small dataset aimed to help users to understand and use the data. It is created by taking the first row from each data leaf, so the number of rows in this Parquet file is the same as the number of data leaves altogether. \par
  This file allows a user to get a quick overview of the whole dataset in the same format as the whole dataset. Given how it is sampled, it will cover the entire width of the dataset and give a user an accurate overview of the properties of the dataset. In such a way, it is superior and more convenient than pointing a user to take out a subset of a single Parquet data leaf for testing. 
    
        \subsubsection{\_metadata and \_common\_metadata} 

   Many parquet reading frameworks support and recommend additional dataset-level metadata files:
   \begin{itemize}
    \item \texttt{\_common\_metadata} which contains the full schema of the dataset, and can be thought of as extensive header information. This fill will know all of the columns and their types, as well as any top-level key-value metadata associated with the full parquet dataset.
    \item \texttt{\_metadata} contains per-partition information, chiefly the footer information of all constituent parquet files which contain aggregate statistics.
\end{itemize}

To understand the value of these files, it is helpful to understand the structure of partitioned parquet files. 
Figure~\ref{fig:partitionedParquet} shows a schematic of two partitioned parquet files. There is some top-level metadata that may describe the full dataset, as well as column-level key-value metadata.
The data values are shown in gray, and typically will take up most of the space of the parquet file. Parquet files may also have footers which contain aggregate statistics about each column (the min value, max value, count of valid values).
\begin{figure}
\centering
\includegraphics[width=0.9\textwidth]{leaf_files.png}
\caption{Example file layout of two parquet files of a dataset}
\label{fig:partitionedParquet}
\end{figure}

For very large datasets, it is helpful to have a single file that holds the common header information in all of the partitioned parquet files (assuming that they have homogeneous structure).
This \texttt{\_common\_metadata} file can be much smaller (and so faster to read) than a leaf parquet file.
The footers, however, will be different for every file, and some files may contain multiple footers if the data inside a single file is very large.
These footers can be concatenated into a single \texttt{\_metadata} file, and can provide valuable insight into the distribution of the data. 
A clever parquet reader can use this information to filter queries to only those partitions where certain values are possible.
See Figure~\ref{fig:parquetMetadata} for the layout of these parquet metadata files.

\begin{figure}
\centering
\includegraphics[width=0.9\textwidth]{metadata_files.png}
\caption{Example file layout of two parquet files of a dataset}
\label{fig:parquetMetadata}
\end{figure}

\subsubsection{collection.properties}\label{sec:collectionProperties}
A text file named \texttt{collection.properties} is REQUIRED for a catalog collection.
It marks the directory as containing a catalog collection, and so MUST be located in the 
root of the catalog collection.
It MUST be encoded in UTF-8, with one line per property, following the syntax \texttt{keyword = value}.
The ordering of the keywords is not important. The keywords MAY include many of those listed in Table~\ref{tab:properties}.

The additional keywords for the \texttt{collection.properties} file to provide linking between catalogs and supplementing tables are:

\begin{itemize}
    \item \texttt{name} human-readable name of the collection
    \item \texttt{hats\_primary\_table\_url} subdirectory of the collection for the primary catalog
    \item \texttt{all\_margins} space-delimited list of margin cache tables
    \item \texttt{default\_margin} the default margin to be used for cross-matching
    \item \texttt{all\_indexes} space-delimited list of index tables
    \item \texttt{default\_index} the default index to be used for id searches (will typically be the survey identifier)
\end{itemize}

\section{Performance Considerations}
Here, we will elaborate on several ways in which this format can be efficiently used. These insights come from our work with LSDB, a Python implementation of a package that works natively with HATS catalogs. \par
\subsection{Parquet storage}
Firstly, we emphasize the need to use the Parquet column filtering. 
This is a standard practice in SQL-like workflows where a user requests only the columns they need, but it is less common in Python-like workflows. 
Loading into memory the columns that a user needs for scientific analysis, usually a couple out of tens or hundreds available, significantly reduced the computational requirements for the analysis.  \par 
Parquet files can also be split into so-called rowgroups. This is the splitting of Parquet into chunks with a fixed number of rows. 
Parquet readers can skip over entire row groups if they don't contain relevant data. 
This can significantly increase the efficiency of particular queries, especially if the rowgroups are selected in a particular way that is appropriate for the scientific case. 
For instance, if rowgroups are made to be small and sorted by the identification number of the survey, the retrieval of the individual rows by survey identification can be made much faster. 
This is because we don't have to load the entire Parquet file into memory, only this tiny rowgroup part, in order to retrieve the needed row from the user. \par 
\subsection{HTTP services}
The fact that the data can be stored on the hard drive and served to the users simplifies the cost structure for catalog providers. 
Still, a user operating on the dataset, even if they are doing aggressive filtering and requesting a minimal number of rows at the end, will have to effectively transfer a large amount of data over to their client, where the filtering is done. 
These limitations could become prohibitive if this is done over a network or with limited bandwidth. To alleviate that problem, it is possible to implement a server-side query in which the filtering operations are done server-side, and only the final dataset is sent to a user. 
Of course, this requires computational resources on the provider's side.
\subsection{Crossmatching}
Finally, we want to highlight the exceptional performance possible when crossmatching HATS catalogs. 
Due to its spatial sharding, the crossmatching approach implemented in LSDB is competitive with the existing tools and is more efficient for extensive catalogs, starting with roughly one million rows. 
Because of the granular spatial structure, the user can increase the number of parallel workers. These will linearly decrease the time needed as long as the number of workers is smaller than the number of partitions in the datasets and there is sufficient I/O speed. 
In general, for typical cases of large catalogs (billion+ rows), crossmatching on a single core is around 5 to 15\% slower than the pure I/O speed. As discussed above, selecting only specific columns and parallelizing the work can drastically improve performance.   
\textcolor{red}{TODO, Add citations } 

    \section{Integration with Existing VO Standards}
    HATS is designed to be compatible with existing VO spatial indexing frameworks, such as HEALPix and MOC (Multi-Order Coverage maps). \par
   \textcolor{red}{TODO, NEED HELP, especially with MOC: explain more how is it compatible}. \par 
   The HATS format can be made to be compatible with the TAP query by implementing a translation layer between the TAP query language. We have explored some initial implementation of such functionality, but the implementation details will always depend on the language used to handle the Parquet files. \par
  We are closely following the development of the VOParquet format and aim to implement it as a part of HATS catalogs.


\appendix
\section{Changes from Previous Versions}
No previous versions yet.

\bibliography{ivoatex/ivoabib,ivoatex/docrepo}

\end{document}

